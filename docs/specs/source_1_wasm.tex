\input source_header.tex

\newcommand{\Rule}[2]{\genfrac{}{}{0.7pt}{}{{\setlength{\fboxrule}{0pt}\setlength{\fboxsep}{3mm}\fbox{$#1$}}}{{\setlength{\fboxrule}{0pt}\setlength{\fboxsep}{3mm}\fbox{$#2$}}}}

\newcommand{\TruE}{\textbf{\texttt{true}}}
\newcommand{\FalsE}{\textbf{\texttt{false}}}
\newcommand{\Rc}{\texttt{\}}}
\newcommand{\Lc}{\texttt{\{}}
\newcommand{\Rp}{\texttt{)}}
\newcommand{\Lp}{\texttt{(}}
\newcommand{\Fun}{\textbf{\texttt{function}}}
\newcommand{\Let}{\textbf{\texttt{let}}}
\newcommand{\Return}{\textbf{\texttt{return}}}
\newcommand{\Const}{\textbf{\texttt{const}}}
\newcommand{\If}{\textbf{\texttt{if}}}
\newcommand{\Else}{\textbf{\texttt{else}}}
\newcommand{\Bool}{\texttt{bool}}
\newcommand{\Number}{\texttt{number}}
\newcommand{\String}{\texttt{string}}
\newcommand{\Undefined}{\texttt{undefined}}

\newtheorem{definition}{Definition}[section]

\begin{document}
	%%%%%%%%%%%%%%%%%%%%%%%%%%%%%%%%%%%%%%%%%%%%%%%
	\docheader{2021}{Source}{\S 1 WebAssembly}{Bernard Teo Zhi Yi}
	%%%%%%%%%%%%%%%%%%%%%%%%%%%%%%%%%%%%%%%%%%%%%%%

\input source_intro.tex

Source \S 1 WebAssembly is the dialect of Source \S 1 used in Sourceror, an experimental compiler from Source \S 1 to WebAssembly.  This dialect is mostly similar to Source \S 1.

\section{Changes compared to Vanilla Source \S 1}

\subsection*{Language}  

All language features of Source \S 1 are supported except for proper tail calls.

Source \S 1 WebAssembly additionally supports proper modules -- it is possible to import external modules from the web, and such modules are also written in Source \S 1 WebAssembly.  The import and export syntax follow that of ECMAScript modules.

\subsection*{Libraries}

The standard libraries (MISC and MATH) are supported with explicit import statements.  There are two differences:  Source \S 1 WebAssembly does not support true varargs, so functions like \lstinline{math_max()} only support up to two arguments; and stringifying a function does not output the literal function body.

Other self-hosted libraries may also be used as long as they are implemented in Source \S 1 WebAssembly.

\subsection*{Error Handling}

Compilation errors produce error messages with line numbers, just like in Source \S 1.  Runtime type errors currently produce error messages without any location information.

\input source_bnf.tex

\input source_1_bnf_with_import_export.tex

\newpage

\input source_return

\input source_import_wasm

\input source_export

\input source_boolean_operators

\input source_names_lang

\input source_numbers

\input source_strings

\input source_comments

\input source_overloading

\input source_typing

\input source_import_syntax

\section{Standard Libraries}

The following libraries are always available in this language.

Names must be imported explicitly before being used.  The MISC library can be imported as "std/misc" and the MATH library can be imported as "std/math".

For example, \lstinline[language=javascript]!import { get_time } from "std/misc";! will import the \lstinline{get_time()} function.

\input source_misc

\input source_math

\input source_js_differences

\end{document}
